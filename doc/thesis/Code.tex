%%%%%%%%%%%%%%%%%%%%%%%%
%%  PROJECT OVERVIEW  %%
%%%%%%%%%%%%%%%%%%%%%%%%

\section{Project Overview}
\label{app:Code}

Accord has gone through many versions, from the first version (0.1)
 in september 2004, to the final (0.9) version. The first version had less than 2000 lines,
 only 6 classes. In between there were several rebuilds of the
 system, from design and up to almost complete implementation, until some major
 design flaw, Java API limitation, and even Java Runtime Error stopped the development.

The final version was a redesign from a February 2005 (0.8) version that incorporated a lot of code from
 earlier versions, but had a redesign (and reimplementation) of the membership
 protocols. The total code lines in Accord amounts to approximately 4400, then including
 the \emph{Accord\-Node\-Test} class, used for bootstrapping Accord system, a GNU Make--file,
 a statistics--calculating script, and a clear--text config file.

Accord consists of five packages each with a set of classes to do some job. Each package and class
 is documented with \emph{Java\-Doc} and is compiled with both \emph{Java\-Doc} to {\tt HTML}, and
 with \emph{Doxygen} to \LaTeX  (then compiled to DVI and converted to PostScript and PDF).

\begin{small}
\renewcommand{\descriptionlabel}[1]%
  {\hspace{\labelsep}\normalsize\tt{#1}}
\begin{description}
\item[net.morimekta.accord] {\em (Accord Core Services)}

     The main services of Accord, following the rules outlined in Section \ref{design:membership}.
  The four main classes are;
  the \emph{Stabilizer}, following the rules outlined in Section \ref{design:stabilization};
  the \emph{I\-Am\-Alive} service, which employs the \emph{keep--alive} protocol for Accord;
  the \emph{Member\-ship} service, which implements the \qmark{join} and \qmark{leave} protocols;
  and the \emph{Lookup} service, managing lookup requests for many of the other services, and
  for other users of Accord.

\item[net.morimekta.accord.tables] {\em (Accord Routing Tables)}

     This package contains the three routing tables of Accord; the \emph{Finger\-Table},
  the \emph{Succ\-List} and the \emph{Pred\-List}. In addition it holds a generalization
  of the routing tables called \emph{Overlay\-Container}, and a collected class combining
  the three tables into a complete routing table system called \emph{Lookup\-Table}.

\item[net.morimekta.net] {\em (Asynchronous Network Communication)}

     On order to achieve asynchronous communication between the nodes in Accord,
  we built an  UDP\cite{udp} based communication protocol. Passing around
  \emph{Message} instances through a \emph{Message\-Socket}, and possibly invoking a
  \emph{Message\-Service}. And
  since nodes in Accord are indexed, a \emph{Location} class (inheriting
  \emph{Inet\-Socket\-Address}) with indexing were built.

\item[net.morimekta.util.index] {\em (Index Utilities)}

   Index utilities that enables arithmetics and dynamic creation and comparison of
  indices. The class \emph{Index} is the index itself with operations for adding, substraction
  and comparison. The class \emph{Index\-Factory} is an abstract factory class for creating indices
  from data or encoded strings. \emph{SHA1\-Factory} is a specialized factory for producing
  SHA1 hashed indices.

\item[net.morimekta.util.std] {\em (Standard Utilities)}

   This package contains four classes that that does arbitrary jobs for Accord, like;
  the class \emph{Log} is an active logger that does formatted timestamps down to the correct millisecond;
  the class \emph{Options} parses and runs an option string with or without argument through an
     abstract method \texttt{parse(c,arg)};
  the class \emph{Config} loads key--value pairs from a config file and stores it into an object
     (or static class);
  and the class \emph{STD} that contains small and simple methods for unsigned big--integer arithmetics
     and various byte--string operations.

\end{description}
\end{small}

Table \ref{tab:ProjectOverview} holds a list of the classes grouped by package with its
 code line count, line count and byte size.

\begin{table} % IKKE [htp] Package and Class overview.
\centering % package.Class | Code lines | Bytes
\small\tt
\begin{tabular}{lrrr}
\bf\em Package and Class &\bf\em Code$^1$ &\bf\em Lines$^2$ &\bf\em Bytes \\ \hline
net.morimekta.accord.AccordNode               &  140 &  298 &   8959 \\
net.morimekta.accord.Conf                     &   27 &   51 &   1641 \\
net.morimekta.accord.IAmAlive                 &  253 &  367 &  13277 \\
net.morimekta.accord.Lookup                   &  462 &  806 &  32929 \\
net.morimekta.accord.Membership               &  972 & 1394 &  61961 \\
net.morimekta.accord.Stabilizer               &  309 &  537 &  22696 \\ \hline
net.morimekta.accord.tables.OverlayContainer  &  185 &  444 &  11830 \\
net.morimekta.accord.tables.PredList          &  131 &  179 &   7096 \\
net.morimekta.accord.tables.SuccList          &  128 &  177 &   6822 \\
net.morimekta.accord.tables.FingerTable       &  126 &  178 &   6378 \\
net.morimekta.accord.tables.LookupTable       &  137 &  275 &   9051 \\ \hline
net.morimekta.net.Location                    &  128 &  299 &   8977 \\
net.morimekta.net.Message                     &  137 &  411 &  10992 \\
net.morimekta.net.MessageService              &    5 &   23 &    684 \\
net.morimekta.net.MessageSocket               &  307 &  627 &  19876 \\ \hline
net.morimekta.util.index.Index                &   75 &  265 &   7280 \\
net.morimekta.util.index.IndexFactory         &   18 &   62 &   1407 \\
net.morimekta.util.index.SHA1Factory          &   53 &   95 &   2455 \\ \hline
net.morimekta.util.std.Config                 &  167 &  303 &  13128 \\
net.morimekta.util.std.Log                    &   80 &  184 &   4669 \\
net.morimekta.util.std.Options                &   99 &  184 &   5287 \\
net.morimekta.util.std.STD                    &  243 &  583 &  17748 \\ \hline
\rm\bf SUM                                    & 4182 & 7742 & 275143 \\ \hline\hline
\end{tabular}
\rm


\parbox{.9\linewidth}{
  \footnotesize
  \begin{enumerate}
  \item Code lines are calculated with the UNIX command:\\
     {\tt \% grep -c "[;\{\}]" \$(find -name *.java | sort) }\\
     from the source code root folder for Accord.
     This is then a number close to the actual lines of \emph{Java code} in the file.
  \item This is the number of \emph{newline characters} in the source file.
  \end{enumerate}
  \small
     The Accord Library (version 0.9) consists of 22 Java classes shown in this table
     distributed over five packages. Each class is documented with \emph{JavaDoc}, authored by
     Stein Eldar Johnsen. The exception is a bundled library class for encoding and
     decoding {\sc Base64} strings which is copyrighted Robert Harder,
     see \underline{http://iharder.net/base64} for more info about the class and its
     properties and development.

     There is also a test-node built for live-testing of the DHT overlay, called
     {\tt Accord\-Node\-Test}, in a package {\tt test}, which is mainly just a bootstrapper
     for an Accord node with options and some lookup functionality and statistical testing.
}
\caption{Class Overview}\label{tab:ProjectOverview}
\end{table}

%%%%%%%%%%%%%%%%%%%%%%
%%  USED APPS INFO  %%
%%%%%%%%%%%%%%%%%%%%%%

% \pagebreak
% \section{Used Applications}
% 
% In the course of the Master a set of applications have been used for development, 
%  development aid, and writing the thesis. Here is a short list and argument for
%  why these applications were chosen.
% 
% \subsection{Java 2.0 SDK}
% 
% The Java 2 Platform, also known as Java SDK 1.5.0, were used as development platform. Before
%  December 2004, Java 1.4 was used as various platforms still dod not have full support for
%  the new version.
% 
% Development in C++ and Python was considered and found more problematic, or more difficult
%  at that stage.Java was chosen because of easiness for development, and access to good
%  Integrated Development Environments (IDEs).
% 
% See {\tt http://www.sun.com/java} for more information on \emph{Java} and \emph{JavaDoc}.
% 
% \subsection{Eclipse}
% 
% Eclipse was used as en IDE for most of the project, and as platform for version control for
%  all text--files used in the thesis. In the beginning in 2003 Eclipse version 2.1 was used, and
%  as soon as 3.0 came, it was substituted. But as the upcoming version 3.1 was going to support
%  Java 2 it was used even as pre-release versions.
% 
% See {\tt http://www.eclipse.org/} for more information on \emph{Eclipse} and its tools.
% 
% \subsection{Kile}
% 
% \LaTeX{} is not an easy typesetting language to learn from scratch. In order to
%  manage the building of the thesis document and its files, and to get usable
%  \emph{debugging} help, Kile was used. Kile is a \LaTeX{} editor and IDE for KDE
%  ({\tt http://www.kde.org/})
% 
% See more information on \emph{Kile} at {\tt http://kile.sourceforge.net/}.
% 
% \subsection{Dia}
% 
% Diagram Editor ............... ........... ........ .......
% 
% See {\tt http://www.gnome.org/projects/dia/} for more on \emph{Dia}.

%%%%%%%%%%%%%%%%%%%
%%  CD ROM INFO  %%
%%%%%%%%%%%%%%%%%%%

\pagebreak
\section{Enclosed CD--ROM Overview}

With this thesis a CD--ROM is enclosed including the source code of the Accord Library,
 its documentation and the thesis document.

\renewcommand{\descriptionlabel}[1]%
  {\hspace{\labelsep}{\tt{#1}}}
\begin{description}
\item[/Accord/src/*] The Accord source code, scripts and default config files.
\item[/Accord/doc/*] Accord Documentation Overview in Java\-Doc HTML format.
\item[/Accord/doc/doxygen/*] Accord documentation compiled with \emph{Doxygen}. With
  PostScript and PDF format files ({\small\tt accord.ps} etc.).
\item[/Thesis/*] Thesis document in PostScript and PDF file format.
\end{description}

Of the content on the CD--ROM only the thesis document is printed.

\emph{Doxygen} is a documentation format like JavaDoc, but understands
 C, C++, Java and more languages, and can generate HTML (like JavaDoc),
 XML Help files, MAN pages (*NIX style) and generate various diagrams
 describing the code and its layout.

% \pagebreak
% 
% \section{Code}
% 
% The code attached.
% 
% \lstset{language=java,basicstyle=\scriptsize,breaklines,numbers=left,numberstyle=\tiny,showstringspaces=false}
% 
% \subsection{net.morimekta.accord}
% \subsubsection{net.morimekta.accord.AccordNode}
% \lstinputlisting{net/morimekta/accord/AccordNode.java}
% \subsubsection{net.morimekta.accord.Conf}
% \lstinputlisting{net/morimekta/accord/Conf.java}
% \subsubsection{net.morimekta.accord.IAmAlive}
% \lstinputlisting{net/morimekta/accord/IAmAlive.java}
% \subsubsection{net.morimekta.accord.Lookup}
% \lstinputlisting{net/morimekta/accord/Lookup.java}
% \subsubsection{net.morimekta.accord.Membership}
% \lstinputlisting{net/morimekta/accord/Membership.java}
% \subsubsection{net.morimekta.accord.Stabilizer}
% \lstinputlisting{net/morimekta/accord/Stabilizer.java}
% \subsection{net.morimekta.accord.tables}
% \subsubsection{net.morimekta.accord.tables.OverlayContainer}
% \lstinputlisting{net/morimekta/accord/tables/OverlayContainer.java}
% \subsubsection{net.morimekta.accord.tables.PredList}
% \lstinputlisting{net/morimekta/accord/tables/PredList.java}
% \subsubsection{net.morimekta.accord.tables.SuccList}
% \lstinputlisting{net/morimekta/accord/tables/SuccList.java}
% \subsubsection{net.morimekta.accord.tables.FingerTable}
% \lstinputlisting{net/morimekta/accord/tables/FingerTable.java}
% \subsubsection{net.morimekta.accord.tables.LookupTable}
% \lstinputlisting{net/morimekta/accord/tables/LookupTable.java}
% \subsection{net.morimekta.net}
% \subsubsection{net.morimekta.net.Location}
% \lstinputlisting{net/morimekta/net/Location.java}
% \subsubsection{net.morimekta.net.Message}
% \lstinputlisting{net/morimekta/net/Message.java}
% \subsubsection{net.morimekta.net.MessageService}
% \lstinputlisting{net/morimekta/net/MessageService.java}
% \subsubsection{net.morimekta.net.MessageSocket}
% \lstinputlisting{net/morimekta/net/MessageSocket.java}
% \subsection{net.morimekta.util.index}
% \subsubsection{net.morimekta.util.index.Index}
% \lstinputlisting{net/morimekta/util/index/Index.java}
% \subsubsection{net.morimekta.util.index.IndexFactory}
% \lstinputlisting{net/morimekta/util/index/IndexFactory.java}
% \subsubsection{net.morimekta.util.index.SHA1Factory}
% \lstinputlisting{net/morimekta/util/index/SHA1Factory.java}
% \subsection{net.morimekta.util.std}
% \subsubsection{net.morimekta.util.std.Config}
% \lstinputlisting{net/morimekta/util/std/Config.java}
% \subsubsection{net.morimekta.util.std.Log}
% \lstinputlisting{net/morimekta/util/std/Log.java}
% \subsubsection{net.morimekta.util.std.Options}
% \lstinputlisting{net/morimekta/util/std/Options.java}
% \subsubsection{net.morimekta.util.std.STD}
% \lstinputlisting{net/morimekta/util/std/STD.java}
% 
% 
% 
