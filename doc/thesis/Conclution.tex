\section{Discussions and Conclusions}
\label{sec:Conclution}
\label{sec:Discussion}

\miniquote{
                                                        A: n. (r$^{oo}$t, rout)\\
1. A road, course, or way for travel from one place to another.\ \ \ \ \ \ \ \ \\
                                 2. A customary line of travel.\ \ \ \ \ \ \ \ \\
                                              B: tr.v. routed, routing, routes \\
                     1. To send or forward by a specific route.\ \ \ \ \ \ \ \ \\
        2. To schedule the order of (a sequence of procedures).\ \ \ \ \ \ \ \ \\
                                                        Dictionary.com/Routing \\
                                                                             \ \\
}

The analysis shows that it is theoretically possible to construct a ring geometry
 DHT that avoids many of the traps of inconsistent routing tables. And the design
 and implementation of Accord helped develop series of remedies, and clean up
 the analysis part. The problem of consistency is not solved
 in general for DHTs, as these remedies
 and solutions must be adapted for each DHT, and there may be DHTs that can not
 use some or any of them.

This section discusses the future of this work, and how it fits with the existing
 DHTs like Chord, Tapestry, Kademlia etcetera. We will also discuss what it really
 solves and how those solutions can be further developed. We do not claim any of
 those solutions are final, and this is a discussion of fitness and relevance only.

\subsection{Comparison and Measures}

Our work was made relative to a ring geometry DHT, but there are many design
 differences between this work and the existing mature DHTs like Chord, Bamboo
 and Tapestry. If the solutions is to be used for any of the other geometries
 the solutions must be adapted, not the underlying DHT.

Here we will try to discuss how the solutions might be adapted for two ring
 geometry DHTs, and then discuss why they cannot be adapted directly for the
 other geometries.

\subsubsection{Chord}

Chord is the base for most of the algorithms in Accord. Both uses a ring geometry
 and skiplist-like routing, and both uses asynchronous messages for working
 with a set of well defined protocols. But
 since Chord and Accord are so similar; are the solutions as valid for Chord as for
 Accord?

\paragraph{Validity}

Both Chord and Accord is a ring geometry DHT, although Accord have \qmark{reversed}
 the ring. So most solutions will work on Chord if predecessors and successors are
 exchanged. In the case of lookup result validity, one have to make a lookup trace
 in Chord and Accord and see the similarity.

When a lookup request in Chord (assume recursive lookup) arrives at a host, it is
 checked for if the node's successor is the index's successor as well, and if so
 it is the node. In Accord it is always forwarded towards its owner until it's a
 direct hit. The difference here is that in Chord, the successor link defines
 the successor's responsibility of indices, and Accord uses the successor to define
 the local node's responsibility.

This can be further investigated by reversing the Accord ring. This makes it the
 \emph{predecessor} that defined local ownership, and, like Chord, the successor
 defines the ownership of it's successor node. If lookups are to be resolved
 at index's owner, the owner must know its exact responsibility range, and thus
 having a correct predecessor. This is similar to Chord in every way except two
 things; responsibility knowledge location (owner node), and lookup replying
 (from the owner, not its predecessor).

From this it can easily be deducted that the lookup consistency solutions can be
 adapted from Accord to Chord, and those DHT's that uses the ring geometry.

\paragraph{Predecessor Resolving Join Protocol}
(PRJP)
In Accord, the base of the \qmark{join protocol} is the successor link resolving
 part (see Section \ref{design:mm-protocols}). If adapted for Chord, this becomes a
 \qmark{predecessor-resolving} join protocol with the successor routing lookup
 procedure. The novel solution is to swap all predecessors with successors
 and visa versa in the algorithm shown in Figure \ref{fig:join} and \ref{fig:joinpred}.

\begin{figure}[htp] % Figure: Chord-Consistency discussion.
\centering
\subfigure[Possibly Inconsistent Tables?]{\label{fig:Chord-IsInconsistent}
\epsfig{file=figures/Chord-IsInconsistent.eps,width=.45\linewidth}}
\subfigure[Consistent Tables in Chord]{\label{fig:Chord-Consistent}
\epsfig{file=figures/Chord-Consistent.eps,width=.45\linewidth}}
% - Possibly inconsistent tables ** (OK)
% - Inconsistent Lookup Trace
% - 
\parbox{.95\linewidth}{ \small
   For Chord this is example of two situation where consistency can be questioned. Figure
    \ref{fig:Chord-IsInconsistent} shows a state where a node has \qmark{joined}, but
    will not receive lookup requests and is not assigned index ownership. Figure
    \ref{fig:Chord-Consistent} shows a situation where a node has \qmark{died} in Chord,
    lookups are consistent, but fails for the dead node's successor.
}
\caption{Consistency in Chord}\label{fig:Chord-Consistency}
\end{figure}

Figure \ref{fig:Chord-Consistency} shows two scenarios in Chord where the routing tables might be
 incorrect yet consistent, and seemingly correct, but may be inconsistent; similar to the
 two situations shown in Figure \ref{fig:incorrect-vs-inconsisten}. The first situation is the
 state of a part of a Chord ring after a node (the one outside the ring) has \qmark{died}.
 Since it is difficult to know wether a node is on the ring, as shown in Figure
 \ref{fig:Chord-IsInconsistent}, Chord needs some protocol to remove this insecurity, which
 could be the source of its inconsistent lookups.

The same can be argued for Bamboo, Pastry and other DHTs with the ring geometry as bottom line.
 But tree geometry DHTs are more reliant on what form of distance metric is used, and how
 this interferes with neighborhood knowledge. If a node in Tapestry doese'nt know any of its two
 immediate neighbors, it will generate inconsistent lookups, although it can still repair it
 self by searching for nodes nearby and \emph{backtrack} to the local node. This still does not
 solve the inconsistency problem.

Since both the \emph{longest prefix} (\emph{XOR}) or \emph{longest suffix} metrics need accurate
 knowledge about \emph{both} immediate neighbors for guaranteed consistent routing on a
 local level, the tree geometry needs a different but similar protocol, which in some
 way guarantees that each neighbor \qmark{gives away} its part of the ownership in a
 two-part commit protocol. This is not discussed further here, as it requires a
 thorough analysis of each failure scenario as for the ring model.

\paragraph{Membership}

The notion of membership is looser in Chord than in Accord, and this has the
 consequence of putting question to what defines membership in Chord? At least two
 definitions of membership can be made in Chord, but both have flaws as it is.

\begin{enumerate}
\item \emph{Routing:}\label{enum:Membership-Routing}
    By defining that being member of the Chord ring is to be
    part of the cycle\cite{liben-nowell-02-observations}, we make the mistake by
    that nodes may after some failure be removed from the \emph{cycle}. There is
    also the problem that a node is still not a member even after completing join,
    that waits until its predecessors in the cycle updates its successor link. This
    is a volatile definition.
\item \emph{Join:}\label{enum:Membership-Join}
    Another way to define membership in Chord is to say that a node
    is a full member the moment the join protocol has notified its successor of it
    being there. This is also problematic, as then routing to the newly joined
    node will fail until its predecessors in the cycle updates its successor link.
    This is a stable definition, although it may be too stable.
\end{enumerate}

Previous studies in Chord have per definition used definition
 \ref{enum:Membership-Routing}, which does not really reflect the true state of the
 DHT, but secures a consistent membership state relative to routing. But in order
 to make membership work \qmark{faster}, we propose to use definition
 \ref{enum:Membership-Join} and employ a predecessor resolving join protocol (PRJP). This
 will ensure that membership is activated at time of join, and offer a consistent
 lookup relevant to that membership definition, and to the consistency definition
 in Table \ref{tab:consistency-definition} in Section \ref{problem:DefiningConsistency}.

\paragraph{Locking Membership Management}

Both the pure predecessor resolving join protocol (PJRP) and the Chord join protocol
 have the problem of concurrent joins. If two nodes try to join with the same
 successor in Chord, or predecessor in Accord, the consistency of the tables are
 at stake. In Chord this only makes a short term \qmark{eviction} of one of
 the joined nodes, which will have to rejoin later.

With the PJRP, four tables at three different nodes will have to be updated
 correctly in order to achieve consistent states at the three consecutive nodes.
 Like with databases, this can be solved by locking concurrent access to the join
 procedures, and to avoid dead-locking a global ordering scheme should be used
 so that the nodes are locked in \emph{the order of the cycle}.

\paragraph{Lookup Protocol in Chord}

When nodes leave in Chord, they just form a \emph{hole} in the ring until the node
 is evicted from its predecessors successor list. One of the problems lies in that
 the lookup protocol, if unsuccessful with contacting a certain node, retries the
 lookup with an \qmark{ignore list} (\cite{chord-homepage}, see source code). which
 includes the unsuccessful node. The new routing will then try the \emph{next}
 successor on the same stage as the failed lookup result, which may at this stage
 have returned (or regained network communication). This adds to the inconsistency
 rating for Chord specifically.

Firstly Chord needs a lookup protocol that does not generate inconsistency with its
 own lookup protocol, then it needs a consistent leave protocol, that can evict nodes
 that don't respond to lookup messages as well as tells of activity (like keep-alive
 protocols).

\paragraph{Leaving Problems}

The argument of Fail-fast nodes are the exact same for Chord as for Accord. If a node
 in Chord looses contact with the rest of the ring for a given time period, the
 rest of the nodes will start removing it from its routing tables, which may cause
 inconsistency with regard to routing.
 
In order to avoid inconsistency, the successor and predecessor should use the leave
 protocol as proposed in Section \ref{design:mm-protocols} (\emph{I-Am-Alive}).
 The commit based leave protocol and the PRJP, Chord should be able to show significantly
 better consistency ratings. This is a topic for further study, and is not discussed
 further in this thesis.

Another problem that comes from nodes returning to the ring before they have been
 completely evicted from the routing tables. In order to avoid this, we propose
 to make each nodes \emph{fail-fast}. That means they must monitor their own
 successfull network activity with
 the ring, and simply shut down if there has been a too long period of inactivity.
 This was discussed in Section \ref{analysis:SingleLinkFailure}, and will not be
 discussed further here.

The tree geometry leave protocol may need special adaptations for guaranteeing that
 the leaving node is not assigned the owner of any index assigned to its neighboring
 node from the leave and onward. This can be easily done since each routing table change
 moves a part of the indices from the leaving node to its neighbor on each side.
 The properties of this depends on the DHT's distance metric,
 and must be analyzed individually.

\subsubsection{Other Ring geometry DHTs}

Now we know there are measures that can work for Chord, but what about the other
 ring geometry DHTs? All the proposals in this thesis (discussion) on consistency
 are shown to work for all ring geometry DHTs that roughly resemble Chord. There is no need
 for other properties than the successor (or predecessor) based routing, so DHTs
 like Bamboo and Pastry will be able to benefit as much.

\subsubsection{Non-ring DHTs}

A question then is; can it work for the other geometries. Not without modifications;
 because
 many of the proposals are dependent on the successive neighbors and global
 ordering as used in routing. Tree geometry DHTs can use various adaptations of the
 protocols, and definitely use the non-protocol specific means proposed in this
 thesis like fail-fast nodes.

\subsection{Other Observations made in the Thesis}

During the project observations not related to consistency have also been made.
 These observations are not thoroughly studies or analyzed, although tested
 for usability and correctness.

\subsubsection{Dynamic Routing Table Preferences}

In the course of the Accord project, it was assumed that the system should not be
 needed to configure the actual sizes of the routing tables. In order to counter
 the need for knowledge about a systems scale, we developed some rules (see
 Section \ref{design:stabilization} and Table \ref{tab:stabilization-rules}) to
 approximate what is optimally needed as the sizes of these tables. Considerations
 of optimality quantities were not made, although these were put in as configurable
 values.

\begin{description}
\item[min\_succ]
     Minimum number of successors. Before any fingers are resolved, ensure
     that this number of consecutive successors are correctly in the routing
     tables.
\item[succ\_ratio]
     Above the number of successors and fingers of \emph{min\_succ}, how many
     of the additional table entries are to be successors (as opposed to fingers).
     This can be used to make an increase in the number of successors above
     \emph{min\_succ} as the number of fingers grows.
\end{description}

What we found was sound \qmark{stabilization rules} that can determine a set of
 successors and fingers to optimize the stability vs. efficiency runtime for each
 node separately.

The rules are not studies beyond the notion of mathematically correctness, and the
 effect of using such rules in a DHT during churn has not been analyzed, simulated
 nor tested and measured. The amount of messages generated by this protocol in
 addition to \emph{normal} network usage is minimal except in the \qmark{minimal}
 state where the number of successors does not reach \emph{min\_succ}.

The rules are only tested to the degree that they work; e.g. they produce the tables
 desired for a given state in the DHT.

\subsubsection{Iteration Restricting Arguments in Lookup/Routing}

It has also been observed that it is simple to combine iterative and recursive
 lookup in a single routing request (see Section \ref{design:lookup}).
 The studies of lookup during churn have been studying solely the differences
 between pure iterative and pure recursive lookup, which \emph{could} be not the
 best options in the world of lookup algorithms.

\subsection{Conclusion}

In this thesis we have shown that active and consistent membership management is
 a possible way of combating inconsistent lookup. With the algorithms and means shown
 here related to consistency built into a ring geometry DHT, consistency should be
 reduced significantly. Not all problems with these protocols have been thoroughly
 studied, which is a work left for further study.

The case for the protocols show they work with simple testing,
 and to produce consistent tables
 both \emph{before} and \emph{after} node join and leave, and analysis show they
 produce consistent lookups even \emph{during} the join process. These protocols
 are mathematically described through state machines, and \emph{should} be possible to
 prove consistent in \emph{all} failure scenarios, but this is left for further study.

\subsection{Further Works}

As the protocols are not actually tested in a live DHT during churn, nor simulated
 with the DHT simulator bundled with Chord\cite{chord-homepage}, this is a job for
 further study directly linked to this work. Testing should be done with an
 implementation of the protocols in a mature DHT to be able to compare results
 with no-membership-management results for consistency rating, CPU and network cost.

Two candidates for this testing are Bamboo, a tree/ring hybrid geometry DHT, and
 Chord, a pure ring DHT. Bamboo has the advantage of being a Java implemented
 DHT, giving the possibility of easier to port. Disadvandages with Bamboo is
 the network event interface, giving disadvantage to both locking and waiting
 protocols.

Chord, the other candidate is thread-based, like Accord, but programmed in C++.
 Chord is also no longer in active development, which makes it a stable platform
 on which to implement the protocols. But when the Accord project was started, it
 was partly because of a set of unnecessary dependencies in Chord combined with
 a not-so-good to missing documentation. Chord is documented good enough to
 develop \emph{for} Chord, but not enough to develop protocols \emph{in} Chord
 without a long pre-study of its inner workings.

The effects of the non-protocol means of reducing inconsistency should also be
 tested or simulated with a live DHT in Churn. Since none of these are dependent
 on a ring geometry DHT, and often require less modification of the original DHT
 they should be easier to add to the DHT, and to test or simulate.

The effect mostly interesting with regard to the different means is their
 effect on; consistency in lookup, lookup latency, network throughput per node
 and CPU use (in case it increases the computational work of nodes significantly).

\paragraph{Dynamic Table Settings}
The \emph{stabilization} protocols developed for Accord should be possible to
 simulate and test further to be able to choose optimal settings for any size
 DHT. This protocol could also be adapted to find optimal settings \emph{not}
 related to routing tables and their sizes. In the case for tree geometry
 DHTs, it must be developed a new scale estimation protocol, which can be used
 for similar purposes.
