%%
%%-----------------------------
%%--                         --
%%--  3  Design of Accord    --
%%--                         --
%%-----------------------------
%%
\section{Design of Accord}
\label{sec:Design}
\miniquote{    V. tr.: To cause to conform or agree; bring into harmony.\ \ \ \ \\
    V. intr.: To be in agreement, unity, or harmony. Synonym [to] agree.\ \ \ \ \\
         N.: Agreement; harmony: act in accord with university policies.\ \ \ \ \\
                                                          Dictionary.com/Accord \\
                                                                              \ \\
}

Accord was started as a small project for testing various methods and consistency
 related protocols on
 a DHT. But after a while it was clear it was becoming quite a complex Java library.
 Basically the Accord system was made to adhere to some usability criteria.

\begin{enumerate}
  \item Platform Independency:
  \begin{enumerate}
    \item By using Java, Accord will be runnable without modification on most platforms today,
      including Windows, UNIX, Linux and Mac OS, though the system design itself it not dependent
      on the system being developed for the Java platform.
    \item Avoiding secondary dependencies by using source based external classes only.
  \end{enumerate}
  \item Distribution:
  \begin{enumerate}
    \item The DHT's theoretical distribution limit is much larger than the physical limits
      of todays networks for practical uses, therefor Accord should not be limited in the
      scale of the distributed system. Other scalability criteria are not explored in this
      design.
    \item By using asynchronous communication, Accord should be very fault tolerant
      on link failures, and each node should be complete runtime independent of its neighbors.
  \end{enumerate}
  \item System performance goals (No specific criteria, this depends on the protocols implemented):
  \begin{enumerate}
    \item Low CPU overhead.
    \item Low Network utilization.
  \end{enumerate}
\end{enumerate}

Accord is basically about agreement of membership. The system is made up
 of four services plus the lookup tables. The routing tables, are built up
 similar to that of Chord\cite{stoica-01-chord}. The four services are;
 the service of looking up nodes in the DHT with the
 \qmark{Lookup Service}; managing joining and leaving of the closest successor
 and predecessor with the \qmark{Membership Manager}; monitoring the status
 of the closest nodes and propagating table information along the ring with the
 \qmark{I-Am-Alive Service}; and \emph{stabilizing} the routing tables with the
 \qmark{Stabilizer}.

The routing tables are described first, and the four services thereafter.

\subsection{Routing in Accord}

Accord is based on the \emph{ring geometry} based on consistent
 hashing \cite{karger-97-consistent-hashing}, with a \emph{skip-list}-like
 routing table structure for efficient lookup
 modelled after Chord \cite{stoica-01-chord}. There are differences between
 Chord and Accord like indexing and the number of routing tables used.

\subsubsection{Indexing}
\label{design:indexing}

Since the objective of Accord was consistency, and not scalability, Accord is designed
 with simplicity rather than efficiency in mind. {\tt SHA-1}\cite{sha-1} is chosen as hash
 algorithm because it is the basis for the Chord algorithm\cite{stoica-01-chord}. Both
 node IDs and data are indexed with {\tt SHA-1}; nodes from the \qstring{ip:port} string,
 and data from its \emph{key} field.

\paragraph{Comparing Indices} Accord was built for being able to use several different
 means of indexing, and a special system of comparing indices was built. Since various
 types and ranges of indices should be comparable, and distributed over the same
 range of values, then both a 4 byte integer and a 40 byte {\tt SHA-1} message digest should
 distribute over an equivalent range of values; \emph{even} if the numerical value of
 the {\tt SHA-1} MD is far greater than
 the maximum value of the integer. This distinction is made to enable Accord to use
 different types of indexing at the same time.

If we align the bits of the two values on the most significant bit (see table
 \ref{tab:byte-align}) and fill the missing bytes with zeroes, we get two numerics
 with approximately the same range, and with equivalent values and value ranges.

\begin{table}[htp]\begin{center}% Index byte alignment.
\begin{tabular}{|ccccccc|}                                                       \hline
byte[19] & byte[18] & byte[17] & byte[16] & byte[15] & ... & byte[0]          \\ \hline
byte[3]  & byte[2]  & byte[1]  & byte[0]  & \multicolumn{3}{|c|}{ ... 000 ...}\\ \hline
\end{tabular}\end{center}
\caption{Index Byte Alignment}
\label{tab:byte-align}
\end{table}

%% ------------------------------------------
\begin{wrapfigure}{r}{.34\linewidth} % fingers-responsibility
\centering
\epsfig{file=figures/Fingers-Responsibility.eps,width=.96\linewidth}
\parbox{.95\linewidth}{
\small
 The various routing tables, including the successor and predecessor lists have
 their index ranges of \qmark{responsibility}.
  }
\caption{Relative Ownership}
\label{fig:responsibility}
\end{wrapfigure}
%% -------------------------------------

\subparagraph{Index Arithmetics} Adding and Subtracting indices are done as aligned and
 padded with zero bytes. The result then has the length of the \emph{largest} index. A
 special method creating an $I_{max}\gg{}n$ value, called \qstring{ImaxRshN(n)} is added
 to the {\em IndexFactory} to create the added offset used to locate the fingers (see
 \ref{design:fingers}).

\paragraph{Distance in Accord}
\label{design:responsibility}
The distance metric used in Accord is the reverse that of Chord. Calculated by
 $d \equiv I_A - I_B (\mathbf{mod}\ I_{max})$ for the distance from $A$ to $B$, and
  it creates a reverse Chord-like ring.

\paragraph{Relative Ownership}
As each node has a limited view of the DHT, each node distributes its immediate neighbors on
 a logical ring, and assigning ranges of responsibility according to that view. The
 logical ring uses the same distance metric, and assigns indices to the closest node.
 The owner of each range of responsibility is determined by the
 \qstring{owner\_of($\tau$)} method.

The range of responsibility for a node $n$ in the ring is set to be
 $\forall\ \tau\ \in\ I_n\ldots(I_{s_1}-1)\ \Leftrightarrow\ n=\mathtt{owner\_of(\tau)}$.

A responsibility range as this has a bad side since there is an absolute
 $I_n\preceq\tau\prec I_{s_1}$ range of responsibility,
 and a worst case scenario a node gets a successor ($s_1$)
 with index $I_n+1$, which gives the node a responsibility of only $1$ index. The good side
 is that it is easier for other nodes to determine the responsible node of a given
 index, as the range has absolute borders defined by the closest known node.

%%%%%%%%%%%%%%%%%%%%%%%%%
%%  OVERLAY STRUCTYRE  %%
%%%%%%%%%%%%%%%%%%%%%%%%%

\subsubsection{Routing Tables}

Accord is a distributed system, and therefore it need a consistent and efficient
 routing table system which with it can do fast searching and maintenance.
 There are three routing tables, the {\em successor list}, {\em predecessor list}
 and {\em finger table}.

The tables are referred to as \qstring{pred}, \qstring{succ} and \qstring{finger}.
 Using \qstring{table[index]}
 for calling the method to fetch the content of one of the routing tables with an
 reference position, similar to what is used in arrays and vectors in C/C++ and Java.
 When referred to with negative table references, e.g. \qstring{finger[-1]}, means to
 count from the end of the list, relative to last filled element, which means that
 \qstring{succ[-2]} means the second last element of the successor list.

\paragraph{Successive Neighbors}
Each node in the ring must know its successor, as it marks the end
 of its responsibility range. It should also know its successor's successor,
 as that is the node that will take over that border if its successor fails. By
 using successive neighbors Accord gains stability, but is required to manage
 more neighbors.

In order to have a consistently updating successor list, each node is responsible of
 \emph{sending its successor list to its predecessor}. This will ensure that when a
 node either enters or leaves the ring, it's membership change knowledge
 will propagate so its predecessor (etc.)
 will include the new knowledge into its successor lists.
 And in order of convenience, the predecessor list is set to be the same size as the
 successor list.

\paragraph{Routing Tables}
\label{design:fingers}
Chord uses two secondary routing tables for achieving efficiency in lookup,
 the \emph{Finger Table} and \emph{Toe Table}. But since each node does not
 have a full knowledge of the number of nodes in the ring, it uses the
 intersection with its successor or predecessor lists as determination of
 the size of the finger table.

Since toes and fingers are somewhat difficult to manage, Accord will only use
 \emph{finger table} and not \emph{toe table}.
 The finger table is a simple skip-list, and does no attempt to
 achieve any lower lookup latency with PRS or PNS.

%%%%%%%%%%%%%%%%%%%%%%
%%  LOOKUP SERVICE  %%
%%%%%%%%%%%%%%%%%%%%%%

\subsection{Lookup Service}
\label{design:lookup}

The lookup service is one of the main reasons for using DHTs, and has been
 discussed in \cite{rhea-04-handling-churn,gupta-04-routing} and more.
 Accord has two lookup procedures. One for looking up directly
 from the overlay tables, which is needed by the \emph{stabilizer}, and one
 for looking up the owner of an index in the ring.

\subsubsection{Routing Table Lookup}
\label{design:table-lookup}

To open the possibility of remote looks on an overlay table, direct table lookups
 are available. Each table lookup is based on the query syntax \qstring{table:index},
 which corresponds to the \qstring{table[idx]} call, although relative indices should
 be allowed here. \qstring{table} is the name of the routing table, and \qstring{index}
 is the table vector index.

\subsubsection{Index Lookup}
\label{design:index-lookup}

Index lookup, or just \qstring{lookup} is a simple operation,
 but can be easily modified with \emph{iterative criteria}.
 Accord uses a no-locking lookup protocol where each external participant
 only checks the request and either forwards or replies to the request.

\paragraph{Partial Result Modes}
 (or just \qmark{mode}).
When looking up an index in the local routing tables, the result will have different
 security properties dependent on where in the tables the node is found. A node can
 be \emph{self}, \emph{neighbor}, \emph{safe} or \emph{unsafe}. The
 \emph{self} mode is when the responding node is the owner of the index, returning
 it self. \emph{neighbor} is when the node is an \emph{immediate} neighbor, either
 \qstring{succ[0]} or \qstring{pred[0]}. The \emph{safe} mode is more complicated;
 if the neighbor lists are marked as \qmark{stable}, then a predecessor is safe, and a successor
 \emph{except the last successor} is called \emph{safe}. All other nodes are called
 \emph{unsafe}.

A routing table is marked \qmark{stable} if there are no changes for a time with the argument
 that nodes that have stayed for some time have a higher propability if staying even
 longer\cite{maymounkov-02-kademlia}.
 The \emph{safe} mode can be argued unnecessary, but the predecessor and successor
 lists are managed differently than the fingers, and the \emph{safe} nodes may be
 the correct owner of an index with much higher probability than the \emph{unsafe}.

\paragraph{Lookup Options}

Lookup behavior in Accord can be modified with some simple options. We can modify
 timeout behavior, try count, and \emph{iteration restriction}. Timeouts in Accord
 are constant but manageable. This is very easy to manage and program, but may
 create some unnecessary timeouts and long waits.

Implementing TCP style or Vivaldi timeout calculation requires monitoring and
 calculating average timeout either for the system as a whole, or for each node.
Constant timeouts was chosen as it does not alter the
 properties of the membership protocols described in Section \ref{design:mm-protocols},
 and are easy to manage.


\subparagraph{Iterative Restrictions}
The \qstring{--iter} option on lookup sets the criteria for whether a node handling a
 lookup request should forward it to the relative owner or reply to the originator of
 the request.
 And there are two sides of the \qstring{--iter} argument; if \qmark{stability} is a
 criteria for iteration {replying to a request}, or if it is a criteria for recursion
 {forwarding of a request}. See table \ref{tab:iter-args} for overview of arguments.

\begin{table}[htp]
\begin{center}
\begin{tabular}{|r|c|l|}                                                                       \hline
{\bf argument    } & {\bf recurse } & {\bf description}                                     \\ \hline
{\tt all         } & {\tt l,n,s,u } & Respond on \emph{all} requests.                       \\ \hline
{\tt safe        } & {\tt l,n,s[u]} & Respond on all requests except on \emph{unsafe} links.\\ \hline
{\tt neighbor    } & {\tt l,n[s,u]} & Respond on local and \emph{neighbor}, else forward.   \\ \hline
{\tt default     } & \              & Use default iterative restriction argument.           \\ \hline
{\tt no-neighbor } & {\tt l[n]s,u } & Respond on all except when node is a \emph{neighbor}. \\ \hline
{\tt no-safe     } & {\tt l[n,s]u } & Respond when the node is \emph{local} or \emph{unsafe}.\\ \hline
{\tt none        } & {\tt l[n,s,u]} & Only respond when node is \emph{self}.                \\ \hline
\end{tabular}
\parbox{.9\linewidth}{ \small
The \qmark{recurse} column shows what behavior to choose on a lookup given its result
 \emph{mode}. The four letters stand for \emph{\underline{l}ocal}, \emph{\underline{n}eighbor},
 \emph{\underline{s}afe} and \emph{\underline{u}nsafe}, and the letters surrounded with
 braces ({\tt[} and {\tt]}) are the modes on which to choose to \qmark{recurse} the request to
 the next node instead of responding to the initiator. The nodes called \emph{neighbor} are the
 immediate neighbors, $s_1$ and $p_1$.
}
\end{center}
\caption{Iterative Restriction Arguments}\label{tab:iter-args}
\end{table}

The \emph{all} and \emph{none} options are equal to those discussed in
 Rhea et al\cite{rhea-04-handling-churn}, called \emph{iterative} and
 \emph{recursive} lookup respectively.

The two options \emph{neighbor} and \emph{safe} are to tell the nodes to; reply on
 immediate neighbors in addition to the local node; and to reply on immediate neighbors
 \emph{and nodes considered safe}, respectively. To call a node \emph{safe} means that
 it is part of a successor or predecessor list that has not been modified for some
 time interval. This is determined by a simple modification \emph{timestamp} which
 is checked against the \emph{current} time.

The two options \emph{no-neighbor} and \emph{no-safe} are opposites of \emph{neighbor}
 and \emph{safe}. They tell to \emph{recurse} on \emph{safer} nodes, and iterate on
 \emph{unsafe} nodes. The case for the \emph{no-neighbor} is that if a node receives a
 lookup request that it believes are bound for its own immediate successor or predecessor
 it can forward it with a high probability of it still being alive.

\subsubsection{Consistent Lookup}
\label{design:consistency-in-lookup}

In Section \ref{sec:Analysis} we discussed consistency with lookup, and found that the
 lookup protocol is good enough as it is, and if the membership protocols make sure the
 routing tables are consistent at \emph{all} times, it does not need to change. We will
 therefor not discuss the matter further.




%%%%%%%%%%%%%%%%%%%%%%%%%%%%%
%%  MEMBERSHIP MANAGEMENT  %%
%%%%%%%%%%%%%%%%%%%%%%%%%%%%%

\subsection{Membership Management}
\label{design:membership}
\label{design:mm-protocols}


In \ref{analysis:MembershipFailure} we showed that inconsistent tables can
 cause inconsistent lookup results. This section looks at how to manage the joining and
 leaving of nodes, and the solutions supposed to solve the inconsistency problem.

We propose two protocols, a new join protocol aimed at updating the routing tables
 in a fashion that produces consistent tables in every step. To guarantee this
 the join protocol is based upon a \emph{2-phase-commit} protocol as showed in
 Ouzo et al\cite{oszu-99-podds}. % OUZO-PODDS: pp 381-390 (383)
 The two \emph{Membership} protocols called \qmark{join} and \qmark{leave} are
 responsible to achieve this.

All the failure cases of \ref{analysis:MembershipFailure} are shown to be handled
 with carefull membership management. There are two important management protocols,
 \qmark{join} and \qmark{leave}. The \emph{join} protocol takes a new node safely into the ring,
 and the \emph{leave} protocol takes a node safely out of the ring.

In addition, the nodes have a push based keep-alive protocol, called the
 \qmark{i-am-alive} service. Since the
 finger table is not reflective, there is no help in propagating it to the
 members. The \emph{stabilizer} is not a protocol per se, but a description of the dynamic
 building and repairing of the routing tables that are used in Accord.


\subsubsection{Neighborhood Responsibility}
\label{design:neigbourhood}

In Accord, each node is responsible for managing it's \emph{successor}, and to notify
 it's \emph{predecessor} of the changes in its successor list and to some degree the
 same for its predecessor. Since each node's
 responsibility is defined as the range from it's own index to it's successors index,
 managing the successor correctly will ensure consistent tables. This stems from that
 each node must keep track of the end of its \emph{responsibility range}
 (\ref{design:responsibility}).

\subsubsection{Join Protocol}

The join operation should be given the ACID properties as data in a database holds.
 When a new member joins the DHT, it should from some moment and on be a member, for
 all operations done thereafter. The four properties have different solutions related
 to the \qmark{join} and \qmark{leave} protocols.

The ACID properties: Atomicity, Consistency, Isolation and Durability are solved with
 2-level 2-phase commit join protocol and commit based leave protocol,
 ordered table changes, membership protocol locking and
 pre-join knowledge respectively. But since some of these means and effects are
 interdependent, they must be assigned to the protocol as a whole.


\begin{figure}[htp]
\centering
\epsfig{file=figures/Join-Sequence-Success.eps,width=\linewidth}
\parbox{.9\linewidth}{
\emph{Scenario:} Successfull \emph{join} using the \qmark{join} and \qmark{joinpred}
 protocols. See figures \ref{fig:join} and \ref{fig:joinpred} for flowchart
 showing the protocol. \emph{Join} in the protocol between $X$ and $Y$, and \emph{joinpred}
 is the protocol between $Y$ and $Z$.
}
\caption{Successfull Join}
\label{fig:join-sequence}
\end{figure}

The join protocol is quite complex, and is described in full in a flowchart in
 Figure \ref{fig:join} and Figure \ref{fig:joinpred}, and as a UML sequence diagram
 for a successfull join operation in Figure \ref{fig:join-sequence}.
Figure \ref{fig:join} shows the interaction between the join protocol initiator (joiner)
 and the joining node's predecessor in the ring, which is the node defining its range
 of responsibility.
Figure \ref{fig:joinpred} shows the interaction with the \emph{coordinator} of the join,
 which is the new node's predecessor, and its successor prior to completion. The flowchart
 shows the three \emph{substates} of the coordinator; \qmark{JP\_INIT}, the \emph{join\-pred
 initial} substate; \qmark{JP\_COM}, the \emph{join\-pred commit} substate; and
 \qmark{JP\_ABO}, the \emph{join\-pred abort} substate.

Note that both protocols follows the two-phase commit in all ways except in termination,
 and in the possibility of \qmark{direct} commit notification if neighbors are already in
 place.

Because of the two-level structure and using consistency as the main criteria for
 termination, the protocol can have a much more relaxed termination protocol, based on
 timeouts, than standard 2PC protocols. The only importance is that the \emph{join-pred}
 protocol does not terminate after sending its commit confirmation.
 This avoids that the confirmation message is lost and the protocol ended,
 thus forcing the predecessor to manually check its successors routing tables.

The ordering of table changes is shown with the \qmark{add} and \qmark{rem} enclosed
 operations. Locking is achieved with locking access to the join and leave protocols
 to only one instance at a time. This ensures that two join operations on the same
 node does not interfere with each other.

\begin{figure}[htp]% Figure: Join flowchart
\centering
\epsfig{file=figures/Join-Flowchart.eps,width=\linewidth}
\parbox{.9\linewidth}{
\begin{enumerate}
{ \small\item Composite boolean operation:\\
  $\neg isIndexConflict\ \wedge\ (\ indexIsMyResponsibility\ \vee\ (joiner = \mathtt{succ[0]}\ )\ )$. }
\end{enumerate}
  Flowchart over the \qmark{join} protocol. The flow ends at the dark gray boxes
   (end states), and the light gray boxes are larger procedures or parts of protocols
   which gives a return value. The protocol follows the basics of the
   two-phase-commit\cite{oszu-99-podds}.}
\caption{\qmark{Node Joining} Protocol Flowchart}
\label{fig:join}
\end{figure}

\paragraph{The Connector-Predecessor \qmark{join} Protocol} is the part going between the
 initiator of the join (the joiner), and the node that is to become its predecessor. The
 flowchart (figure \ref{fig:joinpred}) follows the flow rules of state-machines, and each
 \qmark{box} represent either a state (circle), a decision (diamond), or enclosed
 operations (squares). The dotted and dashed lines are asynchronous messages sent between
 two processes. Light gray boxes represent other state-machines, and darker gray represent
 end states.

The protocol has three main phases; initialization, ready and done. When it is done it
 can either be committed or aborted. In the committed end-state for the initiator node
 it is positively added to the ring. The problem here is if the invoked predecessor fails
 during commit phase. The only way for the initiator to know if join has been successfull
 is to lookup it self in the ring. If response is positive (returns \qmark{self}), the
 node has joined, if the response is negative (other response), the join has failed.

\begin{figure}[htp] % Figure: Join-Pred flowchart
\centering
\epsfig{file=figures/JoinPred-Flowchart.eps,height=0.665\paperheight} \\\ \\
\parbox{.9\linewidth}{
  The \qmark{joinpred} protocol part of the join protocol. The three states
  (\qmark{JP\_INIT}, \qmark{JP\_ABO} and \qmark{JP\_COM})
  are the three states described in \ref{design:mm-protocols}}
\caption{\qmark{Predecessor Joining} Protocol Flowchart}
\label{fig:joinpred}
\end{figure}

\paragraph{The Predecessor-Successor \qmark{joinpred} Protocol} takes part solely between
 the invoked predecessor and its successor (as prior to join).
 On the initiators part (joiner's predecessor) there are three
 states, \emph{joinpred Initiate} \qmark{JP\_INIT}, \emph{joinpred Abort} \qmark{JP\_ABO} and
 \emph{joinpred Commit} \qmark{JP\_COM} as
 showed in the flowchart in figure \ref{fig:joinpred}. In total the three states makes up
 the two phases of the commit protocol. The extra \qmark{loop} on the end of the successors
 protocol makes sure the commit is received by the initiator.

\subsubsection{Leave Protocol}

The leave protocol of Accord is similar to the join protocol, but much simpler. The
 second level of the protocol (between the predecessor and the successor of the
 leaving node) is a one-phase commit protocol, as that removal does not break consistency in
 lookup (see Section \ref{design:consistency-in-lookup} and Section \ref{analysis:MembershipFailure}).
 The important part here is a node's
 recoverability from after being halfway removed from the ring. It is done by invoking
 the join protocol to repair links consistently. See Figure \ref{fig:leave-sequence} for
 sequence of messages during the protocol. No flowchart is given.

\begin{figure}[htp]
\centering
\epsfig{file=figures/Leave-Sequence.eps,width=\linewidth}
\parbox{.9\linewidth}{
\begin{itemize}{\footnotesize
\item The Initiator can either be the leaver it self in case of \qmark{disconnect} or
 the leaver's predecessor in case of failed \qmark{i-am-alive} messages. See Section
 \ref{design:iamalive} for more on that.}
\end{itemize}
{\small
The leave protocol successfull leave protocol use. This case is a mix of the failing
 node and disconnecting scenarios, as on \qmark{disconnect} the node is never
 \emph{checked}, and on failing node it doesn't request a response. }}
\caption{Leave protocol sequence diagram}
\label{fig:leave-sequence}
\end{figure}

Note that the leave protocol only works if the leaver's predecessor can access the
 successor. In case of multiple consecutive node failures, the predecessor of the
 first failing node must start the leave protocol on the last failed node first, and
 then collapse the ring in between.

\subsubsection{I-Am-Alive Protocol}
\label{design:iamalive}

The responsibility of making sure both neighbors know of the continuing activity of
 a node, and propagating neighbor lists throughout the ring is the \qmark{I-Am-Alive
 protocol}.

The I-Am-Alive protocol is designed to take care of updating the predecessor and
 successor lists. It does that by sending messages both ways periodically with its
 continuing lists for the target. So the first successor get sent the predecessor
 list, and the first predecessor gets the successor lists.

Accord uses periodic neighbor update instead of reactive neighbor update because
 periodic updates are simpler to manage, and costs less network activity in case of
 churn. The down side is a constant network traffic cost of two I-Am-Alive messages
 each sending interval for each node in the ring. This is considered trivial as
 Accord is not designed for large scale DHTs.

The I-Am-Alive protocol will activate the \qmark{leave} protocol if it looses contact
 with its successor. But in the case for its predecessor ($p_1$), it will only
 \emph{notify} its $p_2$ when it
 looses contact with its $p_1$. A result of this is that only the predecessor of a
 node can force a node to leave, and only if the node's next alive successor also have removed
 the node. See \ref{design:index-lookup} for proof of consistency on lookup.

%%---------------------------------------
%%--   HENTET FRA DOKUMENTASJONEN !!   --
%%---------------------------------------
%%
%% Counting Rules: (Calculated, not "checked") for a given N:
%% [1] tmp = (( N - min_succ ) * stable_succ_ratio )
%%     FC  = ( tmp > 0 ) ? ( N - min_succ - tmp ) : 0
%% [2] NC  = NC = N - FC
%%
%% No Redundancy rule:
%% [3] FC = 0   : succ[NC] <= pred[NC] < me
%%     FC > 0   : me <= succ[NC] < finger[FC]
%%
%% Completeness Rule:
%% [4] FC = 0   : NC  < min_succ  : pred[NC]          <= succ[NC].succ[0]  < me
%%                NC >= min_succ  : (FH(2)+(Imax>>3)) <= succs[NC]         < me
%%     FC > 0   : FH(FC+1) <= succ[nc] < finger[NC]
%%

\begin{table}[htp]
\centering
\begin{tabular}{|r|c|c|l|}                                                                                  \hline
\multicolumn{4}{|l|}{
Counting Rules:$^1$} \\ \hline
[1] & \multicolumn{3}{|l|}{$tmp = ((N-{\mathtt{min\_succ}})\times{\mathtt{stable\_succ\_ratio}} ) $ }         \\
    & \multicolumn{3}{|l|}{$FC  = \mathtt{if}( tmp > 0 )\mathtt{then}( N - {\mathtt{min\_succ}} - tmp )\mathtt{else}( 0 )$}         \\ 
$\left[2\right]$ & \multicolumn{3}{|l|}{$NC  = N - FC $ }                                                             \\ \hline
\multicolumn{4}{|l|}{Redundancy Avoidance:}                                                              \\ \hline
[3]& $FC=0$ &\multicolumn{2}{|l|}{$(\mathtt{succ}[NC-1]\preceq\mathtt{pred}  [NC-1]\prec\mathtt{me})$}\\
\  & $FC>0$ &\multicolumn{2}{|l|}{$(\mathtt{me}\preceq\mathtt{succ}[NC-1]\prec\mathtt{finger}[FC-1])$}\\ \hline
\multicolumn{4}{|l|}{Completeness:}                                                                      \\ \hline
[4]& $FC = 0$ & $NC<{\mathtt{min\_succ}}$  &
                $(\mathtt{pred}[NC-1]\preceq\mathtt{succ}[NC-1]\mathtt{.succ}[0]\prec\mathtt{me})$     \\
\  &\         & $NC\geq\mathtt{min\_succ}$ & $(BI\preceq\mathtt{succ}[NC-1]\prec\mathtt{me})$          \\ 
\  &$FC\geq 0$& \multicolumn{2}{|l|}{$(FH(FC+1)\preceq\mathtt{succ}[NC-1]\prec\mathtt{finger}[NC-1])$} \\ \hline
\end{tabular}
\parbox{.9\linewidth}{
\begin{enumerate}{\footnotesize
\item These rules are calculated, not \qmark{checked} as a rule that
   can be broken, and the result will vary and is used in the next rules
   to check the actual stability criteria.}
\end{enumerate}
\small
The $(\phi\preceq\tau\prec\psi)$ function is a representation of the
 $between(\phi, \tau, \psi)$ function, notion is similar to the geometrical rule of
 betweenness. $I_{max}$ is the number of values represented in an index, and $BI$ is
 the index marking the \qmark{border} of when to start using fingers. All node references
 refer to their index values unless used for locating other nodes. The $FH(i)$ method
 refers to the \qmark{{\tt{}finger[$i-1$]} is owner of} property. }
\caption{Stabilization Rules}
\label{tab:stabilization-rules}
\end{table}

\subsubsection{Stabilization}
\label{design:stabilization}

Accord uses a periodic routing table updating protocol. The
 difference between periodic and reactive table updates are discussed in
 \cite{rhea-04-handling-churn}. In Accord long periodic update of non-critical tables
 are used, as it requires less monitoring structure like leave notification, but it
 costs some continuous network traffic, and greater risk of outdated tables.

Since the \emph{membership manager} takes care of consistent neighbor joins and leaves,
 and the \emph{i-am-alive} protocol quickly updates the neighborhood lists after such
 updates, only the long range overlay links, and \emph{stabilizing} the lists are left.
 This is the responsibility of the \emph{stabilizer} process.

In Table \ref{tab:stabilization-rules} the method $FH(n)$ is used. It is a function that
 generates the index value which each fingers should be the owner of respectively.

The stabilizer have two main functions:
\begin{enumerate}
\item Make sure all links in the overlay table are valid (alive and a member of the same ring).
\item Stabilize the ratio of successors to fingers according to the current table structure.
\end{enumerate}

To make a simple algorithm for achieving \emph{(2)} we have made some simple rules (see table \ref{tab:stabilization-rules}) that are run periodically to check if the tables are over-extending, insufficient or good. Over-extending tables are reduces, insufficient are increased and good tables are kept as-is.

\subsection{System Design}
\label{degign:SystemDesign}

All in all, there are four main services of Accord: the \emph{Lookup Service}, the \emph{Membership Manager}, the \emph{I-Am-Alive Service} and the \emph{Stabilizer}. These work on a set of \emph{Overlay Containers} which makes up the overlay table or \emph{Lookup Table}.

\begin{figure}[htp] % Accord Class Diagram
\centering
\epsfig{file=figures/Accord-ClassDiagram.eps,width=\linewidth}
\parbox{.9\linewidth}{
  Simple class diagram showing the main functionality of the main classes in Accord.
  The four light gray classes are not part of Accord per se, but are important in
  terms of functionality. The two darker gray classes are standard in Java.}
\caption{Accord Class Diagram}
\label{fig:domainmodel}
\end{figure}
