%%-------------------------
%%--                     --
%%--  1  INTRODUCTION    --
%%--                     --
%%-------------------------
%%
%% 1 - Problem Overview
%% 2 - Terms and abbreviations
%%

\section{Introduction}
\label{sec:introduction}

\miniquote{
    1. A person who has equal standing with another or others, as in rank, class,\\
       or age: children who are easily influenced by their peers.\ \ \ \ \ \ \ \ \\
3. Archaic. A companion; a fellow: \qmark{To stray away into these forests drear,\\
                            /Alone, without a peer} (John Keats).\ \ \ \ \ \ \ \ \\
                                                              Dictionary.com/Peer\\
                                                                               \ \\
}

% \miniquote{
% "A short essay should be like a short skirt; \\
%     Short enough to be interesting, but long \\
%   enough to cover the material"\ \ \ \ \ \ \ \\
%                        - Anonymous Professor \\
% }

With time distributed systems have grown to proportions where the systems stall
 under their own weight. After some scale most distributed systems come to a
 point where adding nodes does not benefit for scale-up not speedup, and in
 the worst case slows down the system services significantly. But a new approach
 with Peer-to-Peer technology may help to solve the problem.

Peer-to-Peer (P2P) technology has many good properties for a distributed
 system. So far mostly used in loosely connected systems like file sharing
 (Gnutella\cite{gnutella-homepage}) and instant messaging
 (Jabber\cite{jabber-homepage}). And a new overlay in P2P is
 Distributed Hash Tables (DHT), which in the beginning were mainly
 projects to create structured P2P search structures, but is now used as basis
 for many different projects.

A Distributed Hash Table is a structure meant to locate the \qmark{owner} of data items
 in a distributed system in a way similar to finding the container block in
 a \qmark{normal} hash table. DHTs have proved to give good qualities in
 terms of efficiency, many using $O(\mathrm{log\ }N)$ hops in
 locating a data owners by using similar degree with routing tables.

\subsection{Problem Background}
\label{problem:overview}

One interesting attribute of Distributed Hash Tables is how simple it is
 to make use of, and be able to rely on it for lookup and routing. And although
 most DHTs report of small to insignificant problems during churn, one problem
 seems still to be unresolved; consistency.

Bamboo tells of $99.99\%$ consistent lookup under light churn (Bamboo F.A.Q.
 \cite{bamboo-homepage}), and $95\%$ under heavy churn. It
 also compares Bamboo to other DHT's mentioned in the thesis, and none of these
 gives $100\%$ consistent lookup. But is it possible? If so it may open more
 possibilities of use for Distributed Hash Tables in general.
 The problem becomes to identify where inconsistency comes from, and finding means
 to counter these situations.

\subsection{Thesis Overview}

In Section \ref{sec:Theory} we will take a look at theory of Peer-to-Peer Networks
 and Distributed Hash Tables. Then in Section \ref{sec:ProblemDescription}
 we outlay
 a detailed problem description, and set out the way we intend to solve the
 problem. The problem is described after the theory because of its close dependency
 on it. Analysis of the various problems are done in Section
 \ref{sec:Analysis}, and
 Section \ref{sec:Design} describes the key parts of the system design. Section
 \ref{sec:Discussion} discusses how the solutions worked with solving the problem,
 and how existing DHTs can make use of it, and what we got out of the analysis and
 the project.

We also append an appendix containing the description of, and an overview over the
 source code of Accord in Appendix \ref{app:Code}. The source code of Accord, it's
 JavaDoc, and thesis PDF file
 are also appended externally with a CD-ROM.
